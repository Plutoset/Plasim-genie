\documentclass[a4paper]{article}
%%%%%%%%%%%%%%%%%%%%%%%%%%%%%%%%%%%%%%%
%%% Setup external packages
%%%%%%%%%%%%%%%%%%%%%%%%%%%%%%%%%%%%%%%
\usepackage[round]{natbib}
\usepackage{amsmath}
\usepackage{graphicx}
%%%%%%%%%%%%%%%%%%%%%%%%%%%%%%%%%%%%%%%
%%% Setup external packages
%%%%%%%%%%%%%%%%%%%%%%%%%%%%%%%%%%%%%%%
\setlength{\textwidth}{6.0in}
\setlength{\textheight}{9.0in}
\setlength{\topmargin}{0.1in}
\setlength{\oddsidemargin}{0.0in}
\newcommand{\gn}{GENIE}
\newcommand{\gnl}{GENIE-land}
\title{\gnl\ documentation version 0.1}
\author{Phil P. Harris}

\begin{document}
\maketitle
\tableofcontents
\newpage
\section{Introduction}\label{s:intro}
This document is the equations that make up the forst version of the
land-surface scheme in \gn.  This scheme is based on Pete Cox's
Interactive Vegetation Model (IVM), which is itself a (very) reduced
version of the tiled UM land-surface scheme MOSES2.  As a result, some
of this text has been grafted from Hadley Centre Technial Notes 24
(TRIFFID) and 30 (MOSES2).

The surface scheme is forced with lowest atmospheric level variables
given in Table~\ref{t:force}.
\begin{table}[hb]
\begin{center}
\begin{tabular}{llll}\hline
Variable & Symbol & Units \\ \hline
Downward solar radiation    & $S_d$ & W~m$^{-2}$ \\
Downward longwave radiation & $L_d$ & W~m$^{-2}$ \\
Air temperature             & $T_1$ & K \\
Air humidity                & $q_1$ & kg~kg$^{-1}$ \\
Air pressure                & $p_1$ & Pa \\
Wind speed                  & $u_1$ & m~s$^{-1}$ \\
Rainfall rate               & $P_r$ & kg~m$^{-2}$~s$^{-1}$ \\
Snowfall rate               & $P_s$ & kg~m$^{-2}$~s$^{-1}$ \\
CO$_2$ concentration        & $p_{CO_2}$ & kg~m$^{-2}$ \\ \hline
\end{tabular}
\caption{Gridbox mean atmospheric variables used to force the land scheme.}\label{t:force}
\end{center}
\end{table}
In each gridbox, the energy, water and carbon balance is calculated
for seven tile types (broadleaf tree, needleleaf tree, C3 grass, C4
grass, shrub, bare soil, land ice) under the same forcing. Gridbox
means (GBMs) of a particular quantity are calculated by weighting tile
values by the fractional coverage of each tile in the gridbox.
e.g. for a quantity $Q$ the GBM is calculated as,
\begin{equation}
  \left< Q \right> = \sum_{i=1}^{ntiles} f_i Q_i, \label{e:gbm}
\end{equation}
where $f_i$ is the fraction of the gridbox covered by tile type $i$.
For much of this document the subscript $i$ is not explicitly written
to make the equations easier to read.  Unless stated otherwise, assume
that an equation is solved for each tile separately.

\section{Radiation}\label{s:radia}
\begin{table}[b]
\begin{center}
\begin{tabular}{lccll}\hline
                       & Symbol        & Units                & Bare soil & Land ice \\ \hline
Snow-free albedo       & $\alpha_{sf}$ & --                   & 0.20      & 0.75 \\
Snow-covered albedo    & $\alpha_{sc}$ & --                   & 0.80      & 0.80 \\
Roughness length       & $z_0$         & m                    & 0.03      & 0.0001 \\
Specific heat capacity & $C_s$         & MJ~K$^{-1}$~m$^{-3}$ & 3.3       & 0.63 \\
Thermal conductivity   & $k_s$         & W~m$^{-1}$~K$^{-1}$  & 0.23      & 0.27 \\ \hline
\end{tabular}
\caption{Albedo and thermal parameters for non-vegetation
tiles.}\label{t:nonveg}
\end{center}
\end{table}

\begin{figure}
\begin{center}
\resizebox{\textwidth}{!}{%
\includegraphics{plots/albedo_lai.eps}%
\includegraphics{plots/albedo_temp.eps}%
\includegraphics{plots/albedo_snow.eps}}
\caption{Albedo of broadleaf tree tiles.  (a)  Snow-free and
  snow-covered albedo as a function of leaf area index, (b) The effect
  of snow aging on snow-covered albedo as a function of tile surface
  temperature, (c) tile albedo as a function of lying snow mass
  (neglecting snow aging effect).}\label{f:alb.bt}
\end{center}
\end{figure}

Snow-free and cold deep snow albedos for non-vegetated tiles
(i.e. bare soil) are given in Table~\ref{t:nonveg}.  For vegetation
tiles with a leaf area index (LAI) of $L$, snow-free and deep snow
albedos are calculated as,
\begin{eqnarray}
  \alpha_{sf} &=& (1 - f_r)\, \alpha_{sf}^{soil} + f_r \, \alpha_v
  \label{e:a0} \\
  \alpha_{sc} &=& (1 - f_r) \, \alpha_v^{smin} + f_r \, \alpha_v^{smax} \label{e:acds}
\end{eqnarray}
where,
\begin{equation}
  f_r = 1- e^{-k\, L} \label{e:fr}
\end{equation}
is the radiative fraction and $k$ is 0.5.  Values of the vegetation
specific parameters $\alpha_v$, $\alpha_v^{smin}$, and
$\alpha_v^{smax}$ are given in Table~\ref{t:veg}.  Snow aging is
represented by reducing the snow albedo when surface temperature,
$T_*$ is greater than $T_m$:
\begin{equation}
  \alpha_s = \left\{
          \begin{array}{lc} 
               \alpha_{sc} & T_* < T_m \\
               \alpha_{sc} + \frac{0.3}{\delta T} \, (\alpha_{sf} -
               \alpha_{sc})(T_* - T_m) & T_m \leq T_* \leq T_m +
               \delta T \\
                0.7\alpha_{sc} + 0.3\alpha_{sf} & T_* > T_m + \delta T
          \end{array}\right. \label{e:als}
\end{equation}
\begin{table}
\caption{There will be an albedo table here.}\label{t:veg}
\end{table}
For a tile in a gridbox with snow mass $S$, the tile albedo
($\alpha_i$) is calculated from,
\begin{equation}
  \alpha_i = \alpha_{sf} + (\alpha_s-\alpha_{sf})\left( 1-e^{-0.2S}\right) \label{e:al}
\end{equation}
This is shown for a broadleaf tree tile in Fig.~\ref{f:alb.bt}c, in
which the effect of snow cover on albedo saturates around
30~kg~m$^{-2}$ snow water equivalent.  Gridbox mean albedos for output
to the atmosphere model are calculated from tile albedos using a
linear weighting by tile fraction (see Eqn.~\eqref{e:gbm}.

Net solar radiation calculated by the atmosphere model is
disaggregated into tile net solar radiation using the tile
and gridbox mean albedos from the previous timestep:
\begin{equation}
  S_{Ni} = S_N \left(\frac{1-\alpha_i}{1-\alpha}\right) \label{e:sn}
\end{equation}


Terrestrial longwave radiation is calculated in the atmosphere module
from the effective radiative surface temperature given by,
\begin{equation}
  T_R = \left(\sum_i f_i T_{*i}^4 \right)^{\frac{1}{4}} \label{e:tr}
\end{equation}

\subsection*{Estimation of PAR}
{\em These are just notes for now.}  The photosynthesis sub-model is
forced by the photosynthetically active portion of the incoming solar
radiation, $I_{par}$, which is approximately half of the total
downward solar radiation flux, $S_d$.  As the atmospheric models in
GENIE have no diurnal cycle of solar radiation the input $S_d$ is a
daily mean value, so the corresponding $I_{par}$ is also a daily mean
value, $\overline{I_{par}}$.  This is insufficient information to
distinguish between long, low maximum $I_{par}$ and short, high
maximum $I_{par}$ days with the same $\overline{I_{par}}$.  This model
uses an additional variable daylength fraction, $D$ (dimensionless),
the fraction of 24 hours when the sun is above the horizon for the
extra information.  Total daily photosynthesis is approximated as a
``square-wave'' such that,
\begin{equation}
  \int_{-\pi}^{\pi} A(I_{par}(Z)) \, dZ \approx A(I_{par}(0)) \, D
\end{equation}
Whereas the maximum daily $I_{par}$, $I_{par}(0)$, is approximated as
a ``triangle'' such that,
\begin{equation}
  I_{par}(0) \approx \frac{2 \, \overline{I_{par}}}{D^{\prime}}
\end{equation}
Where $D^{\prime} = \mbox{max}(D, 0.1)$ is used to prevent
unrealistically high $I_{par}(0)$ values as $D \rightarrow 0$.  Note
that both the total $A$ and $I_{par}(0)$ will usually be overestimated
by these approximations.

\section{Surface fluxes}\label{s:surf}
\subsection{Penman-Monteith derivation}\label{s:surf.penmon}
The surface energy partition for land uses a Penman-Monteith
formulation which relates latent, sensible and ground heat fluxes
directly to the surface radiative energy balance.  Net radiation for a
single tile is calculated as,
\begin{equation}
  R_n = S_{Ni} + L_N \label{e:radbal}
\end{equation}
And this is partitioned into surface heat fluxes as,
\begin{equation}
  R_n = \lambda E + H + G + S \label{e:enbal}
\end{equation}
The fluxes $E$, $H$ and $G$ are expressed using bulk aerodynamic
formulae in terms of the temperature or humidity difference between
the surface and the first atmospheric layer or the sub-surface soil
layer:
\begin{eqnarray}
  E &=& \frac{\psi \rho_*}{r_a} \left[ q_*^s - q_1 \right]
  \label{e:E}\\
  H &=& \frac{\rho_* c_p}{r_a} \left[ T_* - T_1 \right] \label{e:H} \\
  G &=& A_s \left[ T_* - T_s \right] \label{e:G} \\
  S &=& \frac{C_v}{\Delta t} \left[ T_*^{n+1} - T_* \right] \label{e:S}
\end{eqnarray}
where,
\begin{eqnarray}
  \psi &=& \frac{r_a}{r_a+r_s} \label{e:psi}\\
  \rho_* &=& \frac{p_*}{R T_1} \label{e:rho}\\
  A_s &=& \frac{2 k_s}{\Delta z} \label{e:as}
\end{eqnarray}
The thermal inertia of land is provided by a single soil layer, which
has a thermal conductivity $k_s$ and heat capacity $C_s$.  At present,
$k_s$ is not modified by lying snow, when the conductivity of snow is
typically one-third that of soil.  Equations~\eqref{e:radbal}
to~\eqref{e:S} provide six equations for seven unknowns ($R_n$, $T_*$,
$E$, $H$, $G$, $q_*^s$, $T_*^{n+1}$).  A solution is obtained by
linearising the equations in $T_*$ by Taylor expansion of the
non-linear outgoing long-wave and saturated surface humidity about the
lowest atmospheric temperature, $T_1$ to yield,
\begin{eqnarray}
  \sigma T_*^4 &\approx& \sigma T_1^4 + 4\sigma T_1^3(T_*-T_1)
  \label{e:LWlin} \\
  q_*^s &\approx& q_1^s + \alpha_1 \left( T_*-T_1 \right)
  \label{e:QSlin} 
\end{eqnarray}
where $\alpha_1$ is given by the Clausius-Clapeyron equation evaluated
at $T_1$,
\begin{equation}
  \alpha_1 = \left.\frac{\partial q^s}{\partial T} \right|_{T_1}
  =\frac{\epsilon \, \lambda \, q_1^s}{R \, T_1^2} \label{e:a1}
\end{equation}
Substituting Equations~\eqref{e:LWlin} and~\eqref{e:QSlin} into
Equations~\eqref{e:radbal} and~\eqref{e:E}, and using
Equations~\eqref{e:enbal}, \eqref{e:H} and \eqref{e:G} the following
expression for the surface temperature is obtained:
\begin{eqnarray}
  T_*^{n+1} &=& \frac{\left[r_a \hat{A} - \lambda \rho \psi \left(
               \Delta q - \alpha_* T_*^n \right) + \rho c_p T_1 + A_s
               T_{s1} + \frac{C_v}{\Delta t} T_*^n \right]}
               {\left[\alpha_* \lambda \rho \psi + r_a \left( \rho c_p +
               A_s + \frac{C_v}{\Delta t}\right) \right]}\label{e:T*} \\
  \hat{A} &=& S_{Ni} + L_N \label{e:Ahat} \\
  \Delta q &=& q^s(T_*^n) - q_1 \label{e:dq}
%%   T_* &=& T_1 + \left[\frac{r_a \hat{A} - \rho \psi \lambda \Delta q_1}{\rho
%%   \psi \alpha_1 \lambda + \rho c_p + r_a (4\sigma T_1^3 + A_s)}\right]
%%   \label{e:T*} \\
%%   \hat{A} &=& S_N + L_d - \sigma T_1^4 - A_s \left(T_{s1} - T_1\right)
%%   \label{e:Ahat} \\
%%   \Delta q_1 &=& q_1^s - q_1 \label{e:dq}
\end{eqnarray}
This surface temperature is then used in Equations~\eqref{e:E}
and~\eqref{e:H} to calculate the latent and sensible heat fluxes
respectively.  To ensure energy conservation, ground heat flux is
calculated as the residual of the surface energy balance,
\begin{equation}
  G = R_n - \lambda E - H \label{e:G2}
\end{equation}

\subsection{Aerodynamic and surface resistance}\label{s:surf.aero}
Roughness length for bare soil is a constant (see
Table~\ref{t:nonveg}), and for vegetated tiles is calculated as,
\begin{equation}
  z_0 = \Delta_{zh} h \label{e:z0}
\end{equation}
where $\Delta_{zh}$ is a vegetation specific constant and $h$ is the
vegetation height.  The exchange coefficient for sensible and latent
heat fluxes on each tile between the surface and the lowest
atmospheric level at a height $z_1$ is given by,
\begin{equation}
  C_{Hn} = \kappa^2 \left[ \ln\left(\frac{z_1+z_0}{z_0}\right)
  \ln\left(\frac{z_1+z_0}{0.1 z_0}\right)\right]^{-1} \label{e:chn}
\end{equation}
where $\kappa$ is the von Karman constant (0.41).  Aerodynamic
resistance, $r_a$, is calculated using this as,
\begin{equation}
  r_a = \frac{1}{C_{Hn}u_1} \label{e:ga}
\end{equation}
where $u_1$ is the non-directional windspeed at $z_1$.

A soil moisture dependent surface resistance for bare soil is given
by,
\begin{equation}
r_{soil} = \min{ \left( 10^6,\frac{r_{s0}}{\beta} \right) }
%\left\{\begin{array}{lc} 
%                   \frac{r_{s0}}{\beta} & \beta >
%                   \frac{r_{s0}}{10^{-6}} \\ 
%                   10^{-6} & \beta \leq r_{s0}
%                 \end{array}\right.
\label{e:rss}
\end{equation}
where $r_{s0}$ is 100~s~m$^{-1}$.

Surface resistance for vegetation is linked to net leaf
photosynthesis, $A$, following \citet{@@}:
\begin{equation}
  A = \frac{g_c}{1.6 R T_*}\left( c_c - c_i \right) \label{e:ags}
\end{equation}
where $g_c$ is surface conductance (the reciprocal of $r_s$), $R$ is
the ideal gas constant, and $c_c$ and $c_i$ are the leaf surface and
internal CO$_2$ partial pressures respectively.  The factor of 1.6
accounts for the difference between the molecular diffusivities of
water and carbon dioxide.  Leaf photosynthesis is known to be
dependent on a number of environmental variables as well as internal
CO$_2$ concentration, $c_i$:
\begin{equation}
  A = A\left(\underbar{X}, c_i\right) \label{e:a}
\end{equation}
where $\underbar{X}$ represents a general vector of environmental
variables.  Equations~\eqref{e:ags} and~\eqref{e:a} contain three
unknowns: $A$, $g_s$ and $c_i$.  Closure is obtained following the
method of \citet{Jac94}:
\begin{equation}
  \left\{ \frac{c_i - \Gamma}{c_c - \Gamma} \right\} = f_0
  \left\{ 1-\frac{D_*}{D_c}\right\}
\label{e:jac}
\end{equation}

\subsection{Hydrology}\label{s:surf.hyd}
Canopy interception is neglected from the model, so all precipitation
reaches the soil surface even if vegetation is present.  All
precipitation falling as rain infiltrates the soil, and all
precipitation falling as snow is added to the surface lying snow
amount.

Before calculating snowmelt the soil temperature is updated using the
grid-box mean soil heat flux:
\begin{equation}
  T_s^i = T_s^{i-1} + \frac{\Delta t \; G}{C_s \Delta z} \label{e:ts}
\end{equation}
where the indices $i$ and $i-1$ refer to the current and previous
timesteps respectively and $C_s$ is the heat capacity of the soil
(0.33~MJ~K$^{-1}$~m$^{-3}$).  If there is lying snow in the gridbox
and $T_s$ increases from below to above snow melting temperature,
$T_m$ (273.15~K), then $T_s$ is set to melting temperature and the
excess heat flux is used to calculate the rate of snowmelt, $R_m$:
\begin{equation}
  R_m = \frac{k_s \Delta z}{\lambda_{f} \Delta t} \left(T_s^i -
  T_m\right) \label{e:melt}
\end{equation}
Where $\lambda_f$ is the latent heat of fusion (0.334~MJ~kg$^{-1}$).
Note that even if snow is present in a gridbox, the ground heat flux
is used to heat the soil layer before heating the overlying snow.
Lying snow is updated from the rates of snowfall, sublimation and
snowmelt:
\begin{equation}
  M_s^i = \max(M_s^{i-1} + \Delta t (P_s - E_{sub} - R_m),0.0) \label{e:Dsnow}
\end{equation}

Surface and sub-surface runoff rates are calculated using a
``leaky-bucket'' approach.
\begin{eqnarray}
  R_d &=& k_{sat} \left(\frac{M_v}{M_{sat}}
  \right)^{2b+3} \label{e:drain} \\ 
  R_s &=& P + M_s - E \label{e:roff}
\end{eqnarray}
And the soil moisture store updated using,
\begin{equation}
  M_v^i = \max( M_v^{i-1} + \Delta t (P + R_m - E - R_d - R_s), 0.0)
\end{equation}
If soil moisture would become negative during the timestep, then $R_d$
is adjusted to ensure conservation of water.  This means that
sub-surface runoff can be negative - acting, in some ways, like a
water table.

\section{Photosynthesis and respiration}\label{s:photo}
Much of the text in Section~\ref{s:photo} and Section~\ref{s:trif} is
taken verbatim from Hadley Centre Technial Note 24\footnote{Publically
available from
http://www.metoffice.com/research/hadleycentre/pubs/HCTN/HCTN\_24.pdf}.

\subsection{Model Structure} \label{s:photo.stru}
Stomatal openings are the pathways through which both water and carbon
dioxide are exchanged between vegetation and the atmosphere.
Consequently, net leaf photosynthesis, $A$ (mol CO$_{2}$ m$^{-2}$
s$^{-1}$), and stomatal conductance to water vapour, $g_s$ (m
$s^{-1}$), are linked through:
\begin{eqnarray} 
A &=& \frac{g_s}{1.6 \, R \, T_{*}} \,(c_{c} - c_{i})
\label{e:A1}
\end{eqnarray}
where $R$ is the perfect gas constant, $T_{*}$ (K) is the leaf surface
temperature, and $c_{c}$ and $c_{i}$ (Pa) are the leaf surface and
internal CO$_{2}$ partial pressures respectively. The factor of 1.6
accounts for the different molecular diffusivities of water and carbon
dioxide. Leaf photosynthesis is known to be dependent on a number of
environmental variables as well as the internal CO$_{2}$
concentration, $c_{i}$:
\begin{eqnarray}
A &=& A (\vec{X},c_{i})
\label{e:A2}
\end{eqnarray} 
where $\vec{X}$ represents a general vector of environmental
variables.  Equations \eqref{e:A1} and \eqref{e:A2} contain three
unknowns; $A$, $g$ and $c_{i}$. The closure suggested by
\citet{Jac94} is in \gnl\ \citep{Cox98,Cox99}:
\begin{eqnarray}
\left\{ \frac{c_{i}-\Gamma}{c_{c}-\Gamma} \right\}
&=& F_{0} \, \left\{ 1 - \frac{D_{*}}{D_{c}} \right\}
\label{e:A3}
\end{eqnarray}
where $\Gamma$ is the internal partial pressure of CO$_{2}$ at which
photosynthesis just balances photorespiration ( the ``photorespiration
compensation point''), $D_{*}$ is the humidity deficit at the leaf
surface, and $F_{0}$ and $D_{c}$ are vegetation specific parameters
(see Table~\ref{t:gc}).
\begin{table}
\begin{center}
\begin{tabular}{|ll|l|l|l|l|l|} \hline
{\bf Parameter}   &{\bf Units}  & {\bf Broadleaf}& {\bf Needleleaf} & {\bf C$_{3}$ Grass} & {\bf C$_{4}$ Grass} & {\bf Shrub} \\
                  &             & {\bf Tree}     & {\bf Tree}       &&&   \\ \hline
$n_{l} (0)$       & kg N (kg C)$^{-1}$ & 0.040 & 0.030 & 0.060 & 0.030 & 0.030 \\
$\sigma_{l}$      & kg C m$^{-2}$ LAI$^{-1}$     & 0.0375& 0.100 & 0.025 & 0.050 & 0.050 \\
$F_{0}$           &             & 0.875 & 0.875 & 0.900 & 0.800 & 0.900 \\
$D_{c}$           & kg (kg)$^{-1}$ & 0.090 & 0.060 & 0.100 & 0.075 & 0.100 \\
$T_{low}$         & $^{\circ}$C & 0     & -5    & 0     & 13    & 0     \\
$T_{upp}$         & $^{\circ}$C & 36    & 31    & 36    & 45    & 36    \\
\hline
\end{tabular}
\caption{PFT-specific parameters used in the \gnl\ calculation of
vegetation carbon fluxes. The values for top-leaf nitrogen
concentration, $n_{l}(0)$, and specific leaf density, $\sigma_{l}$,
are derived from the survey of \citet{Sch94}, which suggests that
$n_{l}(0) \, \sigma_{l} = 1.5 \times 10^{-3}$ kg N m$^{-2}$ LAI$^{-1}$
for broadleaf plants, and $n_{l}(0) \, \sigma_{l} = 3 \times 10^{-3}$
kg N m$^{-2}$ LAI$^{-1}$ for needleleaf plants.  Values of the maximum
ratio of internal to external CO$_{2}$, $F_{0}$, and the critical
humidity deficit, $D_{c}$, are chosen to give realistic maxima and
humidity dependences for the canopy conductance (see for example,
\citet{Cox98}).  The lower and upper temperatures for photosynthesis,
$T_{low}$ and $T_{upp}$ are consistent with the values prescribed by
\citet{Col91} and \citet{Col92}, except for the introduction of a
finite lower bound for the C$_{3}$ plants, and the shift of the
$V_{m}$ curve for needleleaf trees by -5$^{\circ}$C.}
\label{t:gc}
\end{center}
\end{table}
The leaf photosynthesis models represented by \eqref{e:A2} are based on
the work of \citet{Col91} and \citet{Col92} for C$_{3}$ and C$_{4}$
plants respectively. Details of these models are given below.
However, an additional direct soil moisture dependence is introduced
as suggested by \citet{Cox98}:
\begin{eqnarray}
A &=& A_{p} \, \beta
\label{e:A4}
\end{eqnarray}
where $A_{p}$ is the ``potential'' (non-moisture stressed) rate of net
photosynthesis as given by the models described below, and $\beta$ is
the moisture stress factor:
\begin{equation}
\beta = \left \{ \begin{array}{lll}
              1 & \mbox{for $\Theta > \Theta_{c}$} \nonumber  \\ [5mm]
  \displaystyle{\frac{\Theta-\Theta_{w}}{\Theta_{c}-\Theta_{w}}}
     & \mbox{for $\Theta_{w} < \Theta \leq \Theta_{c}$} \nonumber \\ [5mm]
              0 & \mbox{for $\Theta \leq \Theta_{w}$} \
             \end{array} \right.
\label{e:A5}
\end{equation}
Here, $\Theta_{c}$ and $\Theta_{w}$ are the critical and wilting soil
moisture concentrations respectively, and $\Theta$ is the mean soil
moisture concentration in the rootzone.

Equations \eqref{e:A1} to \eqref{e:A5} represent a coupled model of
stomatal conductance and leaf photosynthesis. Large-scale applications
require an economical means of scaling the predicted leaf-level fluxes
up to the canopy scale. The approach of \citet{Sel92ii} is used here,
in which the primary determinants of photosynthesis, mean incident
photosynthetically active radiation (PAR), $I_{par}$, and the maximum
rate of carboxylation of Rubisco, $V_{max}$, are assumed to be
proportional throughout the plant canopy:
\begin{eqnarray}
I_{par}(l) &=& I_{par}(0) \exp\left\{-k \, l \right\} \label{e:A6} \\ 
V_{max}(l) &=& V_{max}(0) \exp\left\{-k \, l \right\}  \label{e:A7}
\end{eqnarray}
where $(l)$ denotes values beneath $l$ leaf layers, $(0)$ denotes
values at the top of the canopy, and $k=0.5$ is the PAR extinction
coefficient.  This assumption ensures that the relative importance of
each of the photosynthesis limiting factors is the same at every depth
in the canopy. As a consequence it is straightforward to integrate the
leaf conductance and photosynthesis over the canopy leaf area index,
$L$, to yield canopy conductance, $g_{c}$, net canopy photosynthesis,
$A_{c}$, and (non-moisture stressed) canopy dark respiration,
$R_{dc}$:
\begin{eqnarray} 
g_{c} &=& g \, f_{par} \label{e:A8} \\
A_{c} &=& A \, f_{par} \label{e:A9} \\
R_{dc} &=& R_{d} \, f_{par} \label{e:A10}
\end{eqnarray}
where $g$, $A$ and $R_{d}$ are the conductance, net photosynthesis and
(non-moisture stressed) dark respiration rate of the top leaf layer
and
\begin{eqnarray}
f_{par} &=& \frac{1-\exp \left\{-k \, L \right\}}{k}
\label{e:A11}
\end{eqnarray}
Gross primary productivity, $\Pi_{G}$, is equivalent to the gross
canopy photosynthesis:
\begin{eqnarray}
\Pi_{G} &=& 0.012 \, \left\{A_{c} + R_{dc} \, \beta\right\} 
\label{e:A12}
\end{eqnarray}
where the factor $0.012$ converts from units of (mol CO$_{2}$ m$^{-2}$
s$^{-1}$) to (kg C m$^{-2}$ s$^{-1}$), and the second term in the
brackets is the actual (moisture modified) canopy dark
respiration. Net primary productivity, $\Pi$ (kg C m$^{-2}$ s$^{-1}$),
is:
\begin{eqnarray}
\Pi &=& \Pi_{G} - R_{p}
\label{e:A13}
\end{eqnarray}
where $R_{p}$ (kg C m$^{-2}$ s$^{-1}$) is the total plant
respiration. The calculation of $R_{p}$ is described in subsection
\ref{s:photo.resp}.

\subsection{Leaf Photosynthesis Models} \label{s:photo.leaf}
The C$_{3}$ and C$_{4}$ photosynthesis models are based on the work of
\citet{Col91} and \citet{Col92}, as applied by \citet{Sel96}. In both
cases the rate of gross leaf photosynthesis, $W$ (mol CO$_{2}$
m$^{-2}$ s$^{-1}$), is calculated in terms of three potentially
limiting factors:
\begin{enumerate}
\item $W_{c}$ represents the rate of gross photosynthesis when
the photosynthetic enzyme system (RuBP) is limiting:
\begin{equation}
W_{c} = \left \{ \begin{array}{lll}
              \displaystyle{V_{m} \left\{\frac{c_{i} - \Gamma}
 {c_{i} +K_{c} \, (1+O_{a}/K_{o})}\right\}} & \mbox{for C$_{3}$ plants}  \nonumber \\ [5mm]
                 V_{m}                    & \mbox{for C$_{4}$ plants}
             \end{array} \right.
\end{equation}
where $V_{m}$ (mol CO$_{2}$ m$^{-2}$ s$^{-1}$) is the maximum rate of
carboxylation of Rubisco, $O_{a}$ (Pa) is the partial pressure of
atmospheric oxygen, and $K_{c}$ and $K_{o}$ (Pa) are Michaelis-Menten
constants for CO$_{2}$ and O$_{2}$ respectively.
\item $W_{l}$ is the light-limited rate of gross photosynthesis:
\begin{equation}
W_{l} = \left \{ \begin{array}{lll}
              \displaystyle{0.08 \, (1-\omega) \, I_{par} \, \left\{\frac{c_{i} - \Gamma}
 {c_{i} +2 \Gamma}\right\}} & \mbox{for C$_{3}$ plants}  \nonumber \\ [5mm]
                 0.04 \, (1-\omega)  \, I_{par}        & \mbox{for C$_{4}$ plants}
             \end{array} \right.
\end{equation}
where $I_{par}$ is the incident photosynthetically active radiation
(mol PAR photons m$^{-2}$ s$^{-1}$) and $\omega$ is the leaf
scattering coefficient for PAR. The coefficients of 0.08 and 0.04
represent the ``quantum efficiency'' of C$_{3}$ and C$_{4}$ plants
respectively. We follow \citet{Col91} and \citet{Col92} in
assuming $\omega=0.15$ for C$_{3}$ plants, and $\omega=0.17$ for
C$_{4}$ plants.
\item $W_{e}$ is the limitation associated with transport of the
photosynthetic products for C$_{3}$ plants, but is the PEP-Carboxylase
limitation for C$_{4}$ plants \citep{Col92}:
\begin{equation}
W_{e} = \left \{ \begin{array}{lll}
                0.5 \, V_{m} & \mbox{for C$_{3}$ plants}  \nonumber \\ [5mm]
   \displaystyle{2 \times 10^{4} \, V_{m} \, \frac{c_{i}}{p_{*}}}
                             & \mbox{for C$_{4}$ plants}
             \end{array} \right.
\end{equation}
where $p_{*}$ is the surface air pressure.
\end{enumerate}
The actual rate of gross photosynthesis, $W$, is calculated as the
smoothed minimum of these three limiting rates:
\begin{eqnarray}
\beta_{1} W_{p}^{2} - W_{p} \left\{W_{c} + W_{l} \right\} + W_{c} W_{l} &=& 0 \\
\beta_{2} W_{}^{2} - W_{} \left\{W_{p} + W_{e} \right\} + W_{p} W_{e} &=& 0
\end{eqnarray}
where $W_{p}$ is the smoothed minimum of $W_{c}$ and $W_{l}$, and
$\beta_{1}=0.83$ and $\beta_{2}=0.93$ are ``co-limitation''
coefficients. The smallest root of each quadratic is selected. Finally
(non-moisture stressed) net leaf photosynthesis, $A_{p}$, is
calculated by subtracting the rate of dark respiration, $R_{d}$, from
the gross photosynthetic rate, $W$:
\begin{eqnarray}
A_{p} &=& W - R_{d}
\end{eqnarray}
The parameters $R_{d}$, $V_{m}$, $K_{o}$, $K_{c}$ and $\Gamma$ are all
temperature dependent functions derived from \citet{Col91} for
C$_{3}$ plants and \citet{Col92} for C$_{4}$ plants:
\begin{itemize}
\item $V_{m}$, (mol CO$_{2}$ m$^{-2}$ s$^{-1}$) the maximum rate of
carboxylation of Rubisco:
\begin{equation}
V_{m} = \displaystyle{\frac{V_{max} \, f_{T}(2.0)}
{\left[1+\exp \left\{0.3\,(T_{c}-T_{upp}) \right\} \right]
 \left[1+\exp \left\{0.3\,(T_{low}-T_{c}) \right\} \right]}}
\label{e:Vm}
\end{equation}
where $T_{c}$ is the leaf temperature in $^{\circ}$C, $T_{upp}$ and
$T_{low}$ are PFT-dependent parameters, and $f_{T}$ is the standard
``Q10'' temperature dependence:
\begin{eqnarray}
f_{T}(q_{10}) &=& q_{10}^{0.1 \, (T_{c} - 25)}
\end{eqnarray}
The standard photosynthesis models of \citet{Col91} and
\citet{Col92} assume specific values of $T_{upp}$ and $T_{low}$ for
C$_{3}$ and C$_{4}$ plants respectively ($T_{low} \rightarrow
-\infty$, $T_{upp}=36$ $^{\circ}$C for $C_{3}$ plants, and
$T_{low}=13$ $^{\circ}$C, $T_{upp}=45$ $^{\circ}$C for C$_{4}$
plants).  However, in order to capture the temperature responses of
all terrestrial eceosystems, it is necessary to make these parameters
more generally dependent on PFT (i.e. not just dependent on the
photosynthetic pathway). Values of the values chosen are shown in
Table~\ref{t:gc}.

$V_{max}$ (mol CO$_{2}$ m$^{-2}$ s$^{-1}$) is assumed to be linearly
dependent on the leaf nitrogen concentration, $n_{l}$ (kg N (kg
C)$^{-1}$):
\begin{equation}
V_{max} = \left \{ \begin{array}{lll}
             0.0008 \, n_{l} & \mbox{for C$_{3}$ plants}  \nonumber \\ [5mm]
             0.0004 \, n_{l} & \mbox{for C$_{4}$ plants} 
             \end{array} \right.
\end{equation}
The constants of proportionality are derived from \citet{Sch94} by
assuming that dry matter is 40 \% carbon by mass and that the maximum
rate of photosynthesis is approximately equal to $0.5 V_{max}$ for
C$_{3}$ plants and approximately equal to $V_{max}$ for C$_{4}$
plants.
\item $\Gamma$, (Pa) the photorespiration compensation point:
\begin{equation}
\Gamma = \left \{ \begin{array}{lll}
  \displaystyle{\frac{O_{a}}{2 \, \tau}} & \mbox{for C$_{3}$ plants}
                                              \nonumber \\ [5mm]
              0 & \mbox{for C$_{4}$ plants} 
             \end{array} \right.
\end{equation}
where $\tau$ is the Rubisco specificity for CO$_{2}$ relative to
O$_{2}$:
\begin{eqnarray}
\tau &=& 2600 \, f_{T}(0.57)
\end{eqnarray}
\item $K_{c}$ and $K_{o}$ (Pa), Michaelis-Menten constants for
CO$_{2}$ and O$_{2}$:
\begin{eqnarray}
K_{c} &=& 30 \, f_{T}(2.1) \\
K_{o} &=& 3 \times 10^{4} \, f_{T}(1.2)
\end{eqnarray}
\item The rate of dark respiration, $R_d$ (mol CO$_{2}$ m$^{-2}$
s$^{-1}$) is also assumed to have a ``Q10'' temperature dependence,
with a constant of proportionality which depends on $V_{max}$ (i.e.
leaf nitrogen concentration):
\begin{equation}
R_{d} = \left \{ \begin{array}{lll}
                0.015 \, V_{max} \, f_{T}(2.0) & \mbox{for C$_{3}$ plants}  \nonumber \\ [5mm]
                0.025 \, V_{max} \, f_{T}(2.0) & \mbox{for C$_{4}$ plants}  
             \end{array} \right.
\end{equation}
Note: this differs from the dark respiration rate used by \citet{Cox98}
and \citet{Cox99}, which was taken to be directly proportional to
$V_{m}$ as given by \eqref{e:Vm}.
\end{itemize}


\subsection{Plant Respiration} \label{s:photo.resp}
Plant respiration, $R_{p}$, is split into maintenance and growth
respiration:
\begin{eqnarray}
R_{p} &=& R_{pm} + R_{pg}
\end{eqnarray}
Growth respiration is assumed to be a fixed fraction of the net
primary productivity, thus:
\begin{eqnarray}
R_{pg} &=& r_{g} \, \left\{\Pi_{G} - R_{pm} \right\}
\end{eqnarray}
where $\Pi_{G}$ is the gross primary productivity, and the growth
respiration coefficient is set to $r_{g} = 0.25$ for all plant
functional types.  Leaf maintenance respiration is equivalent to the
moisture modified canopy dark respiration, $\beta R_{dc}$, while root
and stem respiration is assumed to be independent of soil moisture,
but to have the same dependences on nitrogen content and
temperature. Thus total maintenance respiration is given by:
\begin{eqnarray}
R_{pm} &=& 0.012 \, R_{dc} \left\{\beta + \frac{(N_{r}+N_{s})}{N_{l}} \right\}
\end{eqnarray}
where $N_{l}$, $N_{s}$ and $N_{r}$ are the nitrogen contents of leaf,
stem and root, and the factor of $0.012$ converts from (mol CO$_{2}$
m$^{-2}$ s$^{-1}$) to (kg C m$^{-2}$ s$^{-1}$). The nitrogen contents
are given by:
\begin{eqnarray}
N_{l} &=& n_{l} \, \sigma_{l} \, L \\
N_{r} &=& \mu_{rl} \, n_{l} \, {\cal R} \\
N_{s} &=& \mu_{sl} \, n_{l} \, {\cal S}
\end{eqnarray}
where $n_{l}$ is the mean leaf nitrogen concentration (kg N (kg
C)$^{-1}$), ${\cal R}$ and ${\cal S}$ are the carbon contents of
respiring root and stem, $L$ is the canopy leaf area index and
$\sigma_{l}$ (kg C m$^{-2}$) is the specific leaf density. The
nitrogen concentrations of root and stem are assumed to be fixed
(functional type dependent) multiples, $\mu_{rl}$ and $\mu_{sl}$, of
the mean leaf nitrogen concentration. In this study, we assume
$\mu_{rl}=1.0$ for all PFTS, $\mu_{sl}=0.1$ for woody plants (trees
and shrubs) and $\mu_{sl}=1.0$ for grasses.  The respiring stemwood is
calculated using a ``pipemodel'' approach in which live stemwood is
proportional to leaf area, $L$, and canopy height, $h$:
\begin{eqnarray}
{\cal S} &=& \eta_{sl} \, h \, L
\label{e:A34}
\end{eqnarray}
The PFT dependent constants of proportionality, $\eta_{sl}$, are
approximated from \citet{Fri93}.

\section{Vegetation dynamics (TRIFFID)}\label{s:trif}
At the core of TRIFFID are first order differential equations
describing how the vegetation carbon density, $C_v$, and fractional
coverage, $\nu$, of a given PFT are updated based on the carbon
balance of that PFT and on competition with other PFTs:
\begin{eqnarray}
\frac{d C_{v}}{dt} &=& (1-\lambda) \, \Pi - \Lambda_{l} \label{e:Cv} \\ [10mm]
C_{v} \, \frac{d \nu}{dt} &=& \lambda \, \Pi \, \nu_{*} \,
\left\{1 - \sum_{j} c_{ij} \, \nu_j \right\}
-\gamma_{\nu} \, \nu_{*} \, C_{v} \label{e:nu}
\end{eqnarray}
\newline
where $\nu_{*}=\mbox{MAX} \left\{\nu,0.01 \right\}$, and $\Pi$ is the
net primary productivity per unit vegetated area of the PFT in
question (as calculated in Section~\ref{s:photo}).  A fraction $\lambda$
of this NPP is utilised in increasing the fractional coverage
(equation \eqref{e:nu}), and the remainder increases the carbon content
of the existing vegetated area (equation \eqref{e:Cv}).  Equation
\eqref{e:Cv} therefore represents the local carbon balance as utilised
in most terrestrial carbon cycle models. TRIFFID is unusual in that
this is coupled to equation \eqref{e:nu}, which is based on the the
Lotka-Volterra approach to intraspecies and interspecies competition
\citep[see for example ]{Sil87}.  Lotka-Volterra equations are used
frequently in theoretical population dynamics but have not previously
been applied in a DGVM. In order to do so here, we have replaced the
usual population state variable of number density with the fractional
area covered by the PFT, and driven increases in $\nu$ directly with
NPP (via the first term on the righthandside of equation
\eqref{e:nu}). Under most circumstances the variable $\nu_*$ is
identical to the areal fraction, $\nu$, but each PFT is ``seeded'' by
ensuring that $\nu_*$ never drops below the ``seed fraction'' of 0.01.

The competition coefficients, $c_{ij}$, represent the impact of
vegetation type ``j'' on the vegetation type of interest (type ``i'',
although for clarity this subscript has been dropped from other
variables in equations \eqref{e:Cv} and \eqref{e:nu}). These
coefficients all lie between zero and unity, so that competition for
space acts to reduce the growth of $\nu$ that would otherwise occur
(i.e. it produces density-dependent litter production). Each PFT
experiences ``intraspecies'' competition, with $c_{ii}=1$ so that
vegetation fraction is always limited to be less than one. Competition
between natural PFTs is based on a tree-shrub-grass dominance
heirachy, with dominant types ``i'' limiting the expansion of
subdominant types ``j'' ($c_{ji}=1$), but not vice-versa ($c_{ij}=0$).
However, the tree types (broadleaf and needleleaf) and grass types
(C$_{3}$ and C$_{4}$) co-compete with competition coefficients
dependent on their relative heights, $h_{i}$ and $h_{j}$:
\begin{eqnarray}
c_{ij}=\frac{1}{1+\exp\left\{20 \, (h_{i}-h_{j})/(h_{i}+h_{j}) \right\}}
\label{e:cocomp}
\end{eqnarray}
The form of this function ensures that the i$^{th}$ PFT dominates when
it is much taller, and the j$^{th}$ PFT dominates in the opposite
limit. The factor of 20 was chosen to give co-competition over a
reasonable range of height differences. Some allowance is made for
agricultural regions, from which the woody types (i.e. trees and
grasses) are excluded, and C$_{3}$ and C$_{4}$ grasses are
reinterpreted as ``crops''.

The $\lambda$ partitioning coefficient in equations \eqref{e:Cv} and
\eqref{e:nu} is assumed to be piecewise linear in the leaf area index,
with all of the NPP being used for growth for small LAI values, and
all the NPP being used for ``spreading'' for large LAI values:
\begin{eqnarray}
\lambda &=& \left \{ \begin{array}{lll}
         1  & \mbox{for $L_b > L_{max}$}   \\ [5mm]
         \displaystyle{\frac{L_{b~~}-L_{min}}{L_{max}-L_{min}}}
     & \mbox{for $L_{min} < L_{b} \leq L_{max}$}  \\ [5mm]
              0 & \mbox{for $L_b \leq L_{min}$}
             \end{array} \right. 
\label{e:lambda}
\end{eqnarray} 
where $L_{max}$ and $L_{min}$ are parameters describing the maximum
and minimum leaf area index values for the given plant functional
type, and $L_b$ is the ``balanced'' LAI which would be reached if the
plant was in ``full leaf''. The actual LAI depends on $L_b$ and the
phenological status of the vegetation type, which is updated as a
function of temperature (see section \ref{s:phenol}).

Changes in vegetation carbon density, $C_v$, are related
allometrically to changes in the balanced LAI, $L_b$.  First, $C_v$ is
broken down into leaf, $\cal{L}$, root, $\cal{R}$, and total stem
carbon, $\cal{W}$:
\begin{eqnarray}
C_v &=& {\cal L} + {\cal R} + {\cal W}
\label{e:Cv2}
\end{eqnarray}
Then each of these components are related to $L_b$. Root carbon is set
equal to leaf carbon, which is itself linear in LAI, and total stem
carbon is related to $L_b$ by a power law \citep{Enq98}:
\begin{eqnarray}
{\cal L} &=& \sigma_l \, L_{b} \label{e:L_LAI}\\ [5mm]
{\cal R} &=& {\cal L} \label{e:R_L} \\ [5mm]
{\cal W} &=& a_{wl} \, L_{b}^{5/3} \label{e:W_L}
\end{eqnarray}
Here $\sigma_l$ is the specific leaf carbon density (kg C m$^{-2}$
LAI$^{-1}$) of the vegetation type, and $a_{wl}$ is a PFT-dependent
parameter in the power law relating LAI and total stem biomass.
Recent work by \citet{Enq98} suggests that 4/3 (rather than 5/3)
may be a more appropriate power to use in the next version of TRIFFID.
Values of canopy height, $h$, are directly from ${\cal W}$ as
described in section \ref{s:sparm}.

The local litterfall rate, $\Lambda_l$, in equation \eqref{e:Cv},
consists of contributions from leaf, root and stem carbon:
\begin{eqnarray}
\Lambda_l &=&\gamma_{l} \, {\cal L} + \gamma_{r} \, {\cal R} + \gamma_{w} \, {\cal W}
\label{e:lambdal}
\end{eqnarray}
where $\gamma_{l}$, $\gamma_{r}$ and $\gamma_{w}$ are turnover rates
(yr$^{-1}$) for leaf, root and stem carbon respectively. The leaf
turnover rate is calculated to be consistent with the phenological
module as described in section \ref{s:phenol}. The root turnover
rate is set equal to the minimum leaf turnover rate $\gamma_{0}=0.25$
for all PFTs, but the total stem turnover is PFT-dependent to reflect
the different fractions of woody biomass (see Table~\ref{t:TRIF}).

There is an additional litter contribution arising from large-scale
disturbance which results in loss of vegetated area at the prescribed
rate $\gamma_{\nu}$, as represented by the last term on the
righthandside of equation \eqref{e:nu}.

\begin{table}[tb]
\begin{center}
\begin{tabular}{|ll|c|c|c|c|c|} \hline
{\bf Parameter}   &{\bf Units}  & {\bf Broadleaf}& {\bf Needleleaf} & {\bf C$_{3}$ Grass} & {\bf C$_{4}$ Grass} & {\bf Shrub} \\
                  &             & {\bf Tree}     & {\bf Tree} &&&   \\ \hline
$a_{wl}$   & kg C m$^{-2}$ & 0.650 & 0.650 & 0.005 & 0.005 & 0.100 \\
$\eta_{sl}$  & kg C m$^{-3}$ & 0.010 & 0.010 & 0.010 & 0.010 & 0.010 \\
$\gamma_{\nu}$ & yr$^{-1}$ & 0.004 & 0.004 & 0.100 & 0.100 & 0.030 \\
$\gamma_{w}$ & yr$^{-1}$   & 0.010 & 0.010 & 0.200& 0.200 & 0.050 \\
$\gamma_{0}$ & yr$^{-1}$   & 0.250 & 0.250 & 0.250 &0.250 & 0.250 \\
$L_{max}$    &             &  9 &  9 & 4 & 4 & 4 \\
$L_{min}$    &             &  3 &  3 & 1 & 1 & 1 \\ \hline 
\end{tabular}
\caption{PFT-specific parameters for the dynamic vegetation component
of TRIFFID.  The values of $a_{wl}$ were chosen to give realistic
maximum biomass densities from equation~\eqref{e:W_L}. The other
parameters were chosen largely by model calibration in offline tests,
but realistic constraints were applied. For example, the large-scale
disturbance rate, $\gamma_{\nu}$, should yield realistic effective
plant lifetimes, and the total stemwood turnover rate, $\gamma_{w}$,
should reflect the differing percentages of wood amongst the PFTs. The
minimum leaf turnover rate, $\gamma_{0}$, was set uniform across the
PFTs for simplicity. This value is also used to specify the turnover
of root biomass.}
\label{t:TRIF}
\end{center}
\end{table}

\subsection{Leaf Phenology} \label{s:phenol}
Leaf mortality rates, $\gamma_{lm}$, for the tree-types are assumed to
be a function of temperature, increasing from a minimum value of
$\gamma_{0}$, as the leaf temperature drops below a threshold value,
$T_{off}$:
\begin{equation}
\gamma_{lm} = \left \{ \begin{array}{lll}
         \gamma_{0} & \mbox{for $T > T_{off}$}  \\ [5mm]
   \gamma_{0} \left\{1 + 9 \, (T_{off}-T) \right\}
     & \mbox{for $T \leq T_{off}$} \\ [5mm]
             \end{array} \right.
\label{e:llit}
\end{equation}
where $T_{off}=0^{\circ}$C for broadleaf trees and $T_{off}=
-30^{\circ}$C for needleleaf trees \citep{Woo87}. The factor of
9 is such that the leaf turnover rate increases by a factor of 10 when
the temperature drops $1^{\circ}$C below $T_{off}$.  Equation
\eqref{e:llit} describes how leaf mortality varies with temperature,
but it is not sufficient to produce realistic phenology.  A new
variable, $p$, is introduced which describes the phenological status
of the vegetation:
\begin{equation}
L = p \, L_{b}
\end{equation}
where $L$ is the actual LAI of the canopy, and $L_{b}$ is the balanced
(or seasonal maximum) LAI as updated by TRIFFID via the inverse of
equation \eqref{e:W_L}.  The phenological status, $p$, is updated on a
daily basis assuming:
\begin{itemize}
\item leaves are dropped at a constant absolute rate ($\gamma_{p} \, L_{b}$) 
when the daily mean value of leaf turnover, as given by equation \eqref{e:llit}, 
exceeds twice its minimum value
\item  budburst occurs at the same rate when $\gamma_{lm}$ drops back below this 
threshold, and ``full leaf'' is approached assymptotically thereafter:
\end{itemize}
\begin{equation}
\frac{dp}{dt} = \left \{ \begin{array}{lll}
         - \gamma_{p} & \mbox{for $\gamma_{lm} > 2 \,\gamma_{0}$}   \\ [5mm]
         \gamma_{p} \left\{1 - p \right\} & \mbox{for $\gamma_{lm} \leq \,2 \, \gamma_{0}$}   \\ [5mm]
             \end{array} \right.
\label{e:p}
\end{equation}
where $\gamma_{p}=20$ yr$^{-1}$. The effective leaf turnover rate,
$\gamma_l$, as used in equation \eqref{e:lambdal}, must also be updated
to ensure conservation of carbon when phenological changes are
occurring:
\begin{equation}
\gamma_{l} = \left \{ \begin{array}{lll}
         - \displaystyle{\frac{dp}{dt}} & \mbox{for $\gamma_{lm} > 2 \,\gamma_{0}$}   \\ [5mm]
          p \, \gamma_{lm} & \mbox{for $\gamma_{lm} \leq \,2 \, \gamma_{0}$}   \\ [5mm]
             \end{array} \right.
\label{e:phenol}
\end{equation}
Taken together, equation \eqref{e:llit}, \eqref{e:p} and \eqref{e:phenol}
amount to a ``chilling-days'' parametrization of leaf phenology.  A
similar approach may be taken for drought-deciduous phenology and for
the cold-deciduous phenology of the other (non-tree) PFTs, but neither
is included in this version of TRIFFID.

\subsection{Soil Carbon}  \label{s:Cs}
Soil carbon storage, $C_s$, is increased by the total litterfall,
$\Lambda_c$, and reduced by microbial soil respiration, $R_s$, which
returns CO$_{2}$ to the atmosphere:
\begin{eqnarray}
\frac{d C_{s}}{dt} &=& \Lambda_{c} - R_{s}
\label{e:Cs}
\end{eqnarray}
In each gridbox, the total litterfall is made-up of the area-weighted
sum of the local litterfall from each PFT (as given by equation
\eqref{e:lambdal}), along with terms due to the large-scale disturbance
rate, $\gamma_{\nu}$, and PFT competition:
\begin{eqnarray}
\Lambda_{c} = \sum_{i} \nu_{i}
\left\{\Lambda_{li}+\gamma_{\nu i} \, C_{vi}+
\Pi_i \, \sum_{j} c_{ij} \, \nu_j \right\} 
\label{e:Lambda_c}
\end{eqnarray}
The competition term (last term on the righthand side of equation
\eqref{e:Lambda_c}) is derived by imposing carbon conservation on the
soil-vegetation system as described by equations \eqref{e:Cv},
\eqref{e:nu} and \eqref{e:Cs}. It implies that the NPP of each PFT will
be lost entirely as litter once the PFT occupies all of the space
available to it (i.e. when $\sum_j c_{ij} \nu_j \, = \, 1 $).

The rate of soil respiration, $R_{s}$, is dependent on the soil
temperature, $T_{s}$, volumetric soil moisture concentration,
$\Theta$, and soil carbon content, $C_{s}$:
\begin{eqnarray}
R_{s} &=& \kappa_{s} \, C_{s} \, f_{\Theta} \, f_{T}
\end{eqnarray}
where $\kappa_{s}=5 \times 10^{-9}$ s$^{-1}$ is the specific soil
respiration rate at 25 $^{\circ}$ C, and $f_{\Theta}$ and $f_{T}$ are
moisture and temperature dependent functions respectively.  The latter
is assumed to take the ``Q10'' form:
\begin{eqnarray}
f_{T} &=& q_{10}^{0.1 \, (T_{s}-25)}
\end{eqnarray}
where $T_{s}$ is the soil temperature in $^{\circ}$C and
$q_{10}=2.0$. The moisture dependence is based on the model of
\citet{McG92} in which the respiration rate increases with soil
moisture content until an optimum value of moisture is
reached. Thereafter the rate of respiration is reduced with further
increases in soil moisture. The curves presented by \citet{McG92}
were approximated by piecewise linear functions in order to minimise
the number of additional soil variables required.
\begin{equation}
f_{\Theta} = \left \{ \begin{array}{lll}
         1 - 0.8 \, \left\{S - S_{o}\right\} & \mbox{for $S > S_{o}$}   \\ [5mm]
  0.2 + 0.8 \, \displaystyle{\left\{\frac{S-S_{w}}{S_{o}-S_{w}}\right\}}
     & \mbox{for $S_{w} < S \leq S_{o}$}  \\ [5mm]
              0.2 & \mbox{for $S \leq S_{w}$} \
             \end{array} \right.
\end{equation}
Here $S$, $S_{w}$ and $S_{o}$ are the (unfrozen) soil moisture, the
wilting soil moisture and the optimum soil moisture as a fraction of
saturation:
\begin{eqnarray}
S &=& \frac{\Theta}{\Theta_{s}} \\[3mm]
S_{w} &=& \frac{\Theta_{w}}{\Theta_{s}} \\ [3mm]
S_{o} &=& 0.5 \, \left\{1 + S_{w} \right\}
\end{eqnarray}
where $\Theta$, $\Theta_{s}$ and $\Theta_{w}$ are the (unfrozen) soil
moisture concentration, the saturation soil moisture concentration and
the wilting soil moisture concentration respectively. 

\subsection{Updating Biophysical Parameters} \label{s:sparm}
In order to close the biophysical feedback loop (see
Figure~\ref{fig:IVM}), the land-surface parameters required by the
\gnl\ are recalculated directly from the LAI and canopy height of each
PFT, each time the vegetation cover is updated.  Values of canopy
height, $h$, are derived by assuming a fixed ratio, $a_{ws}$, of total
stem carbon, ${\cal W}$, to respiring stem carbon, ${\cal S}$:
\begin{eqnarray}
{\cal W} &=& a_{ws} \, {\cal S}
\end{eqnarray}
where we assume $a_{ws}=10.0$ for woody plants and $a_{ws}=1.0$ for
grasses \citep{Fri93}).  Combining with equations \eqref{e:W_L} and
\eqref{e:A34} enables canopy height to be diagnosed directly from the
total stem biomass:
\begin{eqnarray}
h &=& \frac{{\cal W}}{a_{ws} \, \eta_{sl}} \, \left\{\frac{a_{wl}}{{\cal W}}\right\}^{1/b_{wl}}
\end{eqnarray}
The aerodynamic roughness lengths, which are used by \gnl\ to
calculate surface-atmosphere fluxes of heat, water, momentum and
CO$_{2}$, are assumed to be directly proportional to this canopy
height:
\begin{equation}
z_0 = \left \{ \begin{array}{lll}
         0.05 \, h & \mbox{for trees}   \\ [2mm] 
         0.10 \, h & \mbox{for grasses and shrubs}  
\end{array}
\right.
\label{e:z0}
\end{equation}
where $z_0$ is the roughness length for momentum. The roughness
lengths for scalars (heat, water and CO$_{2}$) are taken to be 0.1 of
this value.

The snow-free albedo of each vegetation tile, $\alpha_{sf}$, is
calculated as a weighted sum of the soil albedo, $\alpha_00$, and a
prescribed maximum canopy albedo, $\alpha_{0\infty}$:
\begin{equation}
\alpha_{sf} = \alpha_{00} \, \exp \left\{-k \, L \right\} + \alpha_{0 \infty} \left(1- \exp \left\{-k \, L \right\} \right)
\label{e:albedo}
\end{equation}
where $L$ is the LAI, $k=0.5$ and $\exp \left\{-k L \right\}$
represents the fraction of the incident light which passes through to
the soil surface.  This simple albedo parametrization uses values of
$\alpha_{0 \infty}=0.1$ for tree types, and $\alpha_{0 \infty}=0.2$
for grasses and shrubs. The soil albedo is a geographically varying
field derived from the dataset of \citet{Wil85}.  A similar equation
is used to calculate the ``cold deep-snow'' albedo, but here both
albedo parameters are PFT-dependent. We assume maximum snow albedos of
$\alpha_{s0}=0.3$ for trees, and $\alpha_{s0}=0.8$ for shrubs and
grasses.  The prescribed minimum snow albedos are; $\alpha_{s
\infty}=0.15$ for the tree types, $\alpha_{s \infty}=0.6$ for grass
types and $\alpha_{s \infty}=0.4$ for shrubs.  In all cases these
parameters were chosen to approximate the albedo values used by
\citet{Cox99}.

%% The canopy catchment capacity, $c_{m}$, which determines the amount of
%% water which is freely available for evaporation from the surface,
%% varies linearly with LAI:
%% \begin{equation}
%% c_{m} = 0.5 + 0.05 \, L
%% \label{e:catch}
%% \end{equation}
%% where the offset of $0.5$ represents puddling of water on the soil
%% surface and interception by leafless plants.  The other hydrological
%% land-surface parameters required by \gnl\ are PFT-dependent, but do
%% not depend directly on LAI or canopy height in this version. Root
%% density is taken to fall off exponentially with depth, such that it is
%% $e^{-2}$ of its surface value at a specified rootdepth (of 3.0m for
%% broadleaf trees, 1.0m for needleleaf trees and 0.5m for grasses and
%% shrubs). Roots are assumed to enhance the maximum surface infiltration
%% rate for water by a factor of 4 for trees, and 2 for the other PFTs.

\subsection{Spin-up Methodology}\label{s:TRIF.spin}
Soil carbon and forest area have timescales of order 1000 years to
reach equilibrium which means it is not feasible to carry out this
spin-up in the fully coupled GCM. However, it is still vital to reach
a good approximation to the pre-industrial equilibrium. The
contemporary carbon sink is only a small fraction of the gross carbon
exchanges between the Earth's surface and the atmosphere, and any
significant model drift could easily swamp this signal. With this in
mind, TRIFFID was designed to be usable in both ``equilibrium'' and
``dynamic'' mode.

This flexibility relies on the numerical design of the model. The
TRIFFID equations to update the plant fractional coverage and leaf
area index are written to enable both ``explicit'' and ``implicit''
timestepping. Thus for example, the dynamical equation for leaf area
index, $L$, can be represented by:
\begin{eqnarray}
\frac{d L}{dt} = F(L)
\end{eqnarray}
where $F$ is a non-linear function of $L$. An explicit scheme uses the
beginning-of-timestep value, $L_{n}$, to calculate $F$, whilst a fully
implicit scheme uses the end-of-timestep value, $L_{n+1}$. In general
the update equation may be written:
\begin{eqnarray}
\frac{\Delta L}{\Delta t} = F (L_{n} + f \Delta L)
\end{eqnarray}
where $\Delta t$ is the model timestep and $f$ is the ``forward
timestep weighting factor'', which is 0 for an explicit scheme and 1
for a fully implicit scheme. Taylor expansion about $L_{n}$ provides
an algebraic update for $L$:
\begin{eqnarray}
\Delta L = \frac{F(L_{n}) \, \Delta t}{1 - f \, F^{'}(L_n) \, \Delta t}
\label{e:update}
\end{eqnarray}
where $F^{'}(L_n)$ is the derivative of $F$ with respect to $L$ at
$L=L_{n}$.  For $f=1$ and large timesteps this equation reduces to the
Newton-Raphson algorithm for iteratively approaching the equilibrium
given by $F(L)=0$.

Each of the TRIFFID prognostic equations is written in the form
represented by equation \eqref{e:update}, which allows the model to be
used in two distinct modes.  In ``equilibrium mode'' TRIFFID is
coupled asynchronously to the atmospheric model, with accumulated
carbon fluxes passed from \gnl\ typically every 5 or 10 years.  On
each TRIFFID call, the vegetation and soil variables are updated
iteratively using an implicit scheme ($f=1$) with a long internal
timestep (10,000 years by default). Offline tests have shown that this
approach is very effective in producing equilibrium states for the
slowest variables (e.g. soil carbon and forest cover).  In ``dynamic
mode'', equation \eqref{e:update} is used with $f=0$ and a timestep
equal to the TRIFFID-GCM coupling period (typically 10 days).

Although the equilibrium mode is effective at bringing the slower
components to equilibrium, it is often necessary to carry-out a
subsequent dynamical TRIFFID run so as to allow the faster varying
components (such as grasses) to come into equilibrium with the
seasonally varying climate. During the pre-industrial spin-up of the
HadCM3LC coupled climate-carbon cycle model, \citet{Cox00} completed a
60 year GCM run with TRIFFID in equilibrium mode (5 year coupling
period) and followed this by a GCM simulation of 90 years with TRIFFID
in its dynamical mode (10 day coupling period). This was necessary to
meet the rather stringent requirements of net carbon balance set to
ensure that the current carbon sink was not swamped by model
drift. For many other purposes (such as simulations of
palaeoclimate-vegetation interactions) much shorter simulations should
suffice (e.g. 20 years in equilibrium mode followed by 10 years in
dynamical mode).

\section{Coupling}\label{s:coup}
\subsection{Coupling with GOLDSTEIN ocean}\label{s:coup.gold}
The land and ocean only interact directly through freshwater runoff
from land (see Section~\ref{s:} for details of calculation).  Runoff
from each land point is routed instantaneously to the ocean using an
input field of runoff destinations (specified as ocean points) on the
atmospheric grid (typically either IGCM3 or GOLDSTEIN) with mass
conserving regridding done in {\em genie.F}.  This runoff mask is
derived offline from the model from TRIP $1^{\circ} \times\ 1^{\circ}$
present-day observed river routing.  Example present-day runoff masks
are provided for use with the EMBM and IGCM atmosphere models.

\subsection{Coupling with EMBM atmosphere}\label{s:coup.embm}
\subsection{Coupling with IGCM3 atmosphere}\label{s:coup.igcm}
\subsection{Coupling with genie-fixedatmos atmosphere}\label{s:coup.fixatm}
\subsection{Coupling with GLIMMER ice-sheets}\label{s:coup.glim}
\subsection{Coupling with ATCHEM biogeochemistry}\label{s:coup.chem}

\section{Sample results}\label{s:result}

%%%%%%%%%%%%%%%%%%%%%%%%%%%%%%%%%%%%%%%%%%%%%%%%%%%%%%%%%%%%%%%%%%%%%%
%% REFERENCES
%%%%%%%%%%%%%%%%%%%%%%%%%%%%%%%%%%%%%%%%%%%%%%%%%%%%%%%%%%%%%%%%%%%%%%
\addcontentsline{toc}{part}{\protect References}
\renewcommand{\bibname}{References}
\bibliographystyle{plainnat}
\bibliography{genie_land_refs}
\end{document}
